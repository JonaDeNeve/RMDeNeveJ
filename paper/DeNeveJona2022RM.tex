%==============================================================================
% Sjabloon onderzoeksvoorstel bachelorproef
%==============================================================================%
% Compileren in TeXstudio:
%
% - Zorg dat Biber de bibliografie compileert (en niet Biblatex)
%   Options > Configure > Build > Default Bibliography Tool: "txs:///biber"
% - F5 om te compileren en het resultaat te bekijken.
% - Als de bibliografie niet zichtbaar is, probeer dan F5 - F8 - F5
%   Met F8 compileer je de bibliografie apart.
%
% Als je JabRef gebruikt voor het bijhouden van de bibliografie, zorg dan
% dat je in ``biblatex''-modus opslaat: File > Switch to BibLaTeX mode.

\documentclass{hogent-article}

\usepackage{lipsum} % Voor vultekst

%------------------------------------------------------------------------------
% Metadata over het artikel
%------------------------------------------------------------------------------

%---------- Titel & auteur ----------------------------------------------------

% TODO: (fase 2) geef werktitel van je eigen voorstel op
\PaperTitle{Wat is er nieuw aan Apple's M1 chip en hoe presteert hij tegenover andere computers?}
% Dit is typisch de opdracht en het vak waarvoor dit artikel geschreven is, bv.
% ``Verslag onderzoeksproject Onderzoekstechnieken 2018-2019''
\PaperType{Paper Research Methods: onderzoeksvoorstel}

% TODO: (fase 1) vul je eigen naam in als auteur, geef ook je emailadres mee!
\Authors{Jona De Neve\textsuperscript{1}} % Authors

% Als het hier effectief gaat om een voorstel voor de bachelorproef, dan ben je
% hier verplicht de naam van je co-promotor in te vullen. Zoniet, dan kan je het
% leeg laten.
\CoPromotor{}

% Contactinfo: Geef hier de contactgegevens van elke auteur van het artikel (en
% indien van toepassing ook van de co-promotor).
\affiliation{
    \textsuperscript{1} \href{mailto:jona.deneve.@student.hogent.be}{jona.deneve.@student.hogent.be}
}

%---------- Abstract ----------------------------------------------------------

\Abstract{% TODO: (fase 6)
 Op 10 november 2020 kondigde Apple de nieuwe M1 chip van hun Apple Silicon's aan. Met deze chip gooiden ze hun Mac reeks over een nieuwe boeg. In dit onderzoek wordt onderzocht wat er nieuw aan de chip is en hoe hij verschilt van andere computers. Hiervoor zal worden gekeken hoe hij is opgebouwd, waar welke componenten zich bevinden. Vervolgens zal getest worden hoe hij presteert onder verschillende belastingen. Als laatste zal die data vergeleken worden met andere computers om te zien waar hij in uitblinkt en waar hij tekort komt. Uit de uitslag wordt verwacht dat de M1 chip zoals Apple beweert de beste chip met SoC (System on a chip) voor een laptop of desktop is. Maar dat de beste grafische kaarten en processors die apart in een computer zitten nog altijd voor hoge belastingen, aan een groter energie verbruik, beter zijn.
}

%---------- Onderzoeksdomein en sleutelwoorden --------------------------------
% TODO: (fase 2) Vul de sleutelwoorden aan.

% Het eerste sleutelwoord beschrijft het onderzoeksdomein. Je kan kiezen uit
% deze lijst:
%
% - Mobiele applicatieontwikkeling
% - Webapplicatieontwikkeling
% - Applicatieontwikkeling (andere)
% - Systeembeheer
% - Netwerkbeheer
% - Mainframe
% - E-business
% - Databanken en big data
% - Machineleertechnieken en kunstmatige intelligentie
% - Andere (specifieer)
%
% De andere sleutelwoorden zijn vrij te kiezen.

\Keywords{Hardware; M1 chip; ARM}
\newcommand{\keywordname}{Sleutelwoorden} % Defines the keywords heading name

%---------- Titel, inhoud -----------------------------------------------------

\begin{document}

\flushbottom % Makes all text pages the same height
\maketitle % Print the title and abstract box
    \tableofcontents % Print the contents section
\thispagestyle{empty} % Removes page numbering from the first page

%------------------------------------------------------------------------------
% Hoofdtekst
%------------------------------------------------------------------------------

\section{Inleiding}

% TODO: (fase 2) introduceer je gekozen onderwerp, formuleer de onderzoeksvraag en deelvragen. Wat is de doelstelling (is die S.M.A.R.T.?), wat zal het resultaat zijn van het onderzoek (een Proof-of-Concept, een prototype, een advies, ...)? Waarom is het nuttig om dit onderwerp te onderzoeken?

Wanneer men spreekt over een desktop computer, kun je wel de grote rechthoekige vorm voorstellen. Hierin zitten alle componenten die de computer vormen zoals de grafische kaart, het moederbord. de processor en de RAM. Deze hebben samen met hun koelingssysteem allemaal ruimte nodig binnen de computer. Maar wat als er niet zoveel ruimte nodig is? \\
Met Apple's nieuwe computer de Apple Mac Mini krijg je een voorbeeld van hoe een computer kleiner kan. Deze computer is niet groter dan een brooddoos. Dit komt door Apple Silicon's M1 chip waar de meeste componenten, die anders apart in de computer zitten, ingebouwd zijn. Hiervoor maakt de M1 chip gebruik van een ARM (Advanced RISC Machines) architectuur met SoC (System on a chip). Twee technologieën die al langer werden gebruikt in mobiele telefoons en andere apparaten binnen- en buitenshuis. \\
Over welk effect de chip zal hebben op hedendaagse computers zal in dit onderzoek een antwoord op worden gezocht. Hiervoor zal onderzocht worden hoe de M1 chip is opgebouwd. Vervolgens zullen de prestaties van de M1 chip getest worden. Als laatste zullen deze prestaties vergeleken worden met andere computers, met en zonder SoC's.

\section{Overzicht literatuur}

% TODO: (fase 4) schrijf de literatuurstudie uit en gebruik waar gepast referenties naar de vakliteratuur.

% Refereren naar de literatuur kan met:
% \autocite{BIBTEXKEY} -> (Auteur, jaartal)
% \textcite{BIBTEXKEY} -> Auteur (jaartal)
% Voorbeeld van een referentie waar de auteursnaam geen onderdeel van de zin is~\autocite{Moore2002}.

De M1 chip is Apple Silicon's eerste chip van een nieuwe reeks van SoC chips. Deze verschilt met de andere reeksen omdat hij gemaakt is voor de Mac toestellen die bestaan uit een aantal laptops en desktops. Voor de M1 chip werd in deze toestellen een processor en grafische kaart van Intel gebruikt. Deze componenten zaten allemaal apart op het moederbord en data moest dus lange afstanden afleggen van de ene chip tot de andere. \\
Om dit aan te passen heeft Apple de aparte componenten samen in de M1 chip gestoken. Zo bevat hij dus een processor en grafische kaart met 8 kernen, een Neural Engine met 16 kernen, de RAM en een gemeenschappelijke opslag \autocite{Apple2020}. Dit is mogelijk gemaakt door gebruik te maken van ARM's 'big.LITTLE' architectuur. Dit betekent dat de chip processor kernen heeft die kleine taken en andere die grote taken afwerken \autocite{ARM2014}. Daardoor kan energie verbruik verminderd worden omdat de processors gemaakt om grote taken uit te voeren niet ook kleine taken moeten verwerken. \\
Overschakelen van Intel's chip naar de M1 chip betekent ook dat Apple van een CISC (Complex Instruction Set Computer) architectuur naar één met RISC verandert. Het verschil tussen deze twee architecturen is dat RISC tegenover CISC smalle instructies van een vaste lengte gebruikt, één instructie per klokcyclus heeft en de druk legt op efficiënt aantal cyclussen per instructie \autocite{Thornton2018}. Deze overstap betekent volgens \textcite{Frazelle2021} betere prestaties, minder energie verbruik en lagere warmte creatie voor de M1 chip. Daarnaast heeft, volgens \textcite{Dalakoti2022} die de chips vergeleken, de M1 chip een betere cross-compatibility tussen mobile apparaten en pc's dankzij de portabiliteit van de chip. \\
Het veranderen van architectuur kwam wel met een groot probleem. Software die gemaakt was voor de Mac met Intel kon niet zomaar starten op de nieuwe Mac's met M1 chips. Hiervoor moest Apple dus een oplossing voor zoeken. Zo kwam Rosetta 2.0 tot stand. Dit programma gebruikt dynamische binaire vertaling om x86\_64 code om te zetten naar Arm64 architectuur \autocite{Edwards2022}.

\section{Methodologie}

% TODO: (fase 5) beschrijf in detail in welke fasen je onderzoek uiteenvalt, hoe lang elke fase duurt en wat het concrete resultaat van elke fase is. Welke onderzoekstechniek ga je toepassen om elk van je onderzoeksvragen te beantwoorden? Gebruik je hiervoor experimenten, vragenlijsten, simulaties? Je beschrijft ook al welke tools je denkt hiervoor te gebruiken of te ontwikkelen.

In dit onderzoek zal eerst onderzocht worden hoe Apple's M1 chip is opgebouwd. Hiervoor zal een literatuur onderzoek gehouden worden om de infrastructuur en de blauwdruk van de chip te vinden. Hierop kunnen we vervolgens zien waar welke componenten zich op de chip bevinden, hoeveel ruimte elk op de chip inneemt en hoe ze met elkaar zijn verbonden. Dit kunnen we dan ook vergelijken met hoe de Mac voor de M1 opgebouwd was. \\
Vervolgens zullen de prestaties van de M1 chip getest worden. Zo zal gekeken worden naar hoeveel stroom de chip verbruikt, hoe goed de taken worden afgewerkt en de status van de chip na deze taken onder verschillende belastingen. Eerst na het opstarten van de computer wanneer er geen applicaties open staan. Dan met de software die de meeste mensen op hun computer gebruiken zoals de Office programma's,  teleconferentie applicaties en browsers. Als laatste onder hoge belasting zoals het spelen van video games, het verwerken van video's of artificiële intelligentie. \\
Als laatste wordt de prestaties vergeleken met andere computers op de markt. Zowel van die met SoC's als degene zonder. Zo wordt gekeken waar de M1 chip voordelen biedt en waar deze tekort komt. Daarnaast wordt hij ook vergeleken met de laatste Mac met de Intel onderdelen.

\section{Verwachte conclusies}

% TODO: (fase 6) beschrijf wat je verwacht uit je onderzoek en waarom (bv. volgens je literatuuronderzoek is softwarepakket A het meest gebruikte en denk je dat het voor deze casus ook het meest geschikt zal zijn). Natuurlijk kan je niet in de toekomst kijken en mag je geen alternatieve mogelijkheden uitsluiten. In de praktijk gebeurt het ook vaak dat een onderzoek tot verrassende resultaten leidt, dat maakt het proces nog interessanter!

Er wordt verwacht dat Apple Silicon's M1 chips een nieuwe soort computers op de markt zal brengen. Dit wordt mogelijk gemaakt door de ARM-architectuur die voor een compacte structuur zorgt. Hierdoor kan data sneller van het ene component naar het andere en door de gemeenschappelijke opslag moet data niet steeds gekopieerd worden. Daarnaast ziet het ernaar uit dat de chips energie zuiniger zijn dankzij ARM's big.LITTLE. \\
Als we \textcite{Apple2020} mogen geloven zou de M1 chip tot wel 3,5 keer sneller zijn dan andere processors en zou het de snelste ingebouwde grafische kaarten hebben. Zo is de chip dus in vergelijking met andere computers een grote stap vooruit maar Computers met de nieuwste grafische kaarten en processors zullen, hoewel deze meer energie en ruimte nodig hebben, hogere belastingen kunnen verdragen. Daarom zullen mensen die zware taken op hun computer moeten uitvoeren waarschijnlijk nog steeds een computer zonder SoC nemen. \\
Maar dit zal niet het laatste zijn dat we over deze SoC chips in laptops of desktops horen. In juni 2022 kondigde Apple weer een nieuwe en verbeterde chip aan, de M2 chip. Zo lijkt het dat SoC potentie heeft om de toekomst van de computer te zijn.

%------------------------------------------------------------------------------
% Referentielijst
%------------------------------------------------------------------------------
% TODO: (fase 4) de gerefereerde werken moeten in BibTeX-bestand
% bibliografie.bib voorkomen. Gebruik JabRef om je bibliografie bij te
% houden.

\phantomsection
\printbibliography[heading=bibintoc]

\end{document}
